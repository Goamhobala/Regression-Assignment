
\documentclass[12pt,halfline,a4paper,]{ouparticle}

% Packages I think are necessary for basic Rmarkdown functionality
\usepackage{hyperref}
\usepackage{graphicx}
\usepackage{listings}
\usepackage{xcolor}
\usepackage{fancyvrb}
\usepackage{framed}

% Link coloring
\hypersetup{breaklinks=true,
            bookmarks=true,
            pdfauthor={},
            pdftitle={STA2005S - Regression Assignment}
            }


%% To allow better options for figure placement
%\usepackage{float}

% Packages that are supposedly required by OUP sty file
\usepackage{amssymb, amsmath, geometry, amsfonts, verbatim, endnotes, setspace}

% use upquote if available, for straight quotes in verbatim environments
\IfFileExists{upquote.sty}{\usepackage{upquote}}{}

% Macros for dealing with affiliations, footnotes, etc.
\makeatletter
\def\Newlabel#1#2#3{\expandafter\gdef\csname #1@#2\endcsname{#3}}

\def\Ref#1#2{\@ifundefined{#1@#2}{???}{\csname #1@#2\endcsname}}

\newcommand*\samethanks[1][\value{footnote}]{\footnotemark[#1]}

\newcommand*\ifcounter[1]{%
  \ifcsname c@#1\endcsname
    \expandafter\@firstoftwo
  \else
    \expandafter\@secondoftwo
  \fi
}

\newcommand*\thanksbycode[1]{%
  \ifcounter{FNCT@#1}
    {\samethanks[\value{FNCT@#1}]}
    {\thanks{\Ref{FN}{#1}}\newcounter{FNCT@#1}\setcounter{FNCT@#1}{\value{footnote}}}
}

% Create labels for Addresses if the are given in Elsevier format

% Create labels for Footnotes if the are given in Elsevier format

% Part for setting citation format package: natbib

% Part for setting citation format package: biblatex

% Pandoc syntax highlighting
\usepackage{color}
\usepackage{fancyvrb}
\newcommand{\VerbBar}{|}
\newcommand{\VERB}{\Verb[commandchars=\\\{\}]}
\DefineVerbatimEnvironment{Highlighting}{Verbatim}{commandchars=\\\{\}}
% Add ',fontsize=\small' for more characters per line
\usepackage{framed}
\definecolor{shadecolor}{RGB}{248,248,248}
\newenvironment{Shaded}{\begin{snugshade}}{\end{snugshade}}
\newcommand{\AlertTok}[1]{\textcolor[rgb]{0.94,0.16,0.16}{#1}}
\newcommand{\AnnotationTok}[1]{\textcolor[rgb]{0.56,0.35,0.01}{\textbf{\textit{#1}}}}
\newcommand{\AttributeTok}[1]{\textcolor[rgb]{0.13,0.29,0.53}{#1}}
\newcommand{\BaseNTok}[1]{\textcolor[rgb]{0.00,0.00,0.81}{#1}}
\newcommand{\BuiltInTok}[1]{#1}
\newcommand{\CharTok}[1]{\textcolor[rgb]{0.31,0.60,0.02}{#1}}
\newcommand{\CommentTok}[1]{\textcolor[rgb]{0.56,0.35,0.01}{\textit{#1}}}
\newcommand{\CommentVarTok}[1]{\textcolor[rgb]{0.56,0.35,0.01}{\textbf{\textit{#1}}}}
\newcommand{\ConstantTok}[1]{\textcolor[rgb]{0.56,0.35,0.01}{#1}}
\newcommand{\ControlFlowTok}[1]{\textcolor[rgb]{0.13,0.29,0.53}{\textbf{#1}}}
\newcommand{\DataTypeTok}[1]{\textcolor[rgb]{0.13,0.29,0.53}{#1}}
\newcommand{\DecValTok}[1]{\textcolor[rgb]{0.00,0.00,0.81}{#1}}
\newcommand{\DocumentationTok}[1]{\textcolor[rgb]{0.56,0.35,0.01}{\textbf{\textit{#1}}}}
\newcommand{\ErrorTok}[1]{\textcolor[rgb]{0.64,0.00,0.00}{\textbf{#1}}}
\newcommand{\ExtensionTok}[1]{#1}
\newcommand{\FloatTok}[1]{\textcolor[rgb]{0.00,0.00,0.81}{#1}}
\newcommand{\FunctionTok}[1]{\textcolor[rgb]{0.13,0.29,0.53}{\textbf{#1}}}
\newcommand{\ImportTok}[1]{#1}
\newcommand{\InformationTok}[1]{\textcolor[rgb]{0.56,0.35,0.01}{\textbf{\textit{#1}}}}
\newcommand{\KeywordTok}[1]{\textcolor[rgb]{0.13,0.29,0.53}{\textbf{#1}}}
\newcommand{\NormalTok}[1]{#1}
\newcommand{\OperatorTok}[1]{\textcolor[rgb]{0.81,0.36,0.00}{\textbf{#1}}}
\newcommand{\OtherTok}[1]{\textcolor[rgb]{0.56,0.35,0.01}{#1}}
\newcommand{\PreprocessorTok}[1]{\textcolor[rgb]{0.56,0.35,0.01}{\textit{#1}}}
\newcommand{\RegionMarkerTok}[1]{#1}
\newcommand{\SpecialCharTok}[1]{\textcolor[rgb]{0.81,0.36,0.00}{\textbf{#1}}}
\newcommand{\SpecialStringTok}[1]{\textcolor[rgb]{0.31,0.60,0.02}{#1}}
\newcommand{\StringTok}[1]{\textcolor[rgb]{0.31,0.60,0.02}{#1}}
\newcommand{\VariableTok}[1]{\textcolor[rgb]{0.00,0.00,0.00}{#1}}
\newcommand{\VerbatimStringTok}[1]{\textcolor[rgb]{0.31,0.60,0.02}{#1}}
\newcommand{\WarningTok}[1]{\textcolor[rgb]{0.56,0.35,0.01}{\textbf{\textit{#1}}}}

% tightlist command for lists without linebreak
\providecommand{\tightlist}{%
  \setlength{\itemsep}{0pt}\setlength{\parskip}{0pt}}

% From pandoc table feature
\usepackage{longtable,booktabs,array}
\usepackage{calc} % for calculating minipage widths
% Correct order of tables after \paragraph or \subparagraph
\usepackage{etoolbox}
\makeatletter
\patchcmd\longtable{\par}{\if@noskipsec\mbox{}\fi\par}{}{}
\makeatother
% Allow footnotes in longtable head/foot
\IfFileExists{footnotehyper.sty}{\usepackage{footnotehyper}}{\usepackage{footnote}}
\makesavenoteenv{longtable}


\usepackage{float} \floatplacement{figure}{H} \usepackage{caption} \captionsetup[figure]{font=scriptsize} \renewenvironment{abstract}{}{} \renewenvironment{keywords}{}{}
\usepackage{booktabs}
\usepackage{longtable}
\usepackage{array}
\usepackage{multirow}
\usepackage{wrapfig}
\usepackage{float}
\usepackage{colortbl}
\usepackage{pdflscape}
\usepackage{tabu}
\usepackage{threeparttable}
\usepackage{threeparttablex}
\usepackage[normalem]{ulem}
\usepackage{makecell}
\usepackage{xcolor}

\begin{document}

\title{STA2005S - Regression Assignment}

\author{%
%
% Code for old style authors field
%
% Add \and if both authors and author
%
%
% Code for new (elsevier) style author field
\name{Jing Yeh}
%
\email{\href{mailto:yhxjin001@myuct.ac.za}{yhxjin001@myuct.ac.za}}%
%
%
%
\and
\name{Saurav Sathnarayan}
%
\email{\href{mailto:sthsau001@myuct.ac.za}{sthsau001@myuct.ac.za}}%
%
%
%
%
}

\abstract{}

\date{2024-10-16}

\keywords{}

\maketitle



\newpage

\hypertarget{part-one-analysis}{%
\subsection{Part One : Analysis}\label{part-one-analysis}}

\hypertarget{section-1-introduction}{%
\section{Section 1: Introduction}\label{section-1-introduction}}

Air pollution, particularly high levels of particulate matter (PM), is a
major environmental and public health issue in South Africa's urban
centers. Exposure to elevated PM levels is linked to respiratory
diseases and other serious health conditions. Understanding the factors
influencing PM concentrations is crucial for developing policies that
improve air quality and protect public health. This analysis seeks to
identify the key drivers of air pollution in South Africa's cities,
focusing on how various urban, environmental, and socioeconomic factors
affect particulate matter levels.\\
Unknown Factors to Investigate:\\
Traffic Density: How do varying levels of vehicle traffic contribute to
PM levels in different areas?\\
Industrial Activity: What is the impact of industrial activity near
monitoring stations on air quality?\\
Temperature \& Humidity: How do changes in weather conditions, like
temperature and humidity, influence PM concentrations?\\
Wind Speed: How does wind speed affect the dispersion or accumulation of
particulate matter in urban areas?\\
Day of the Week \& Public Holidays: Do patterns of human activity on
weekdays, weekends, and holidays significantly influence pollution
levels?\\
Urban Greenery: How effective are green spaces in reducing air pollution
in densely populated areas?

\hypertarget{objective}{%
\section{Objective}\label{objective}}

\hfill\break
The goal of this analysis is to explore the relationships between PM
levels and these explanatory variables. By identifying the most
influential factors, we aim to inform urban planning and public health
strategies that address air pollution and improve the quality of life in
South African cities.

\hypertarget{section-2-data-exploration}{%
\subsection{Section 2 : Data
Exploration}\label{section-2-data-exploration}}

density plot

pairwsie plots

\begin{Shaded}
\begin{Highlighting}[]
\NormalTok{continuous\_vars }\OtherTok{\textless{}{-}}\NormalTok{ data\_tidy\_air\_quality[, }\FunctionTok{sapply}\NormalTok{(data\_tidy\_air\_quality, is.numeric)]}
\FunctionTok{pairs}\NormalTok{(continuous\_vars, }\AttributeTok{main =} \StringTok{"Pairwise Scatterplots of Continuous Variables"}\NormalTok{)}
\end{Highlighting}
\end{Shaded}

\includegraphics{Report_files/figure-latex/unnamed-chunk-1-1.pdf}

categorial variable plots

\begin{Shaded}
\begin{Highlighting}[]
\NormalTok{data\_tidy\_air\_quality}\SpecialCharTok{$}\NormalTok{industrial\_activity }\OtherTok{\textless{}{-}} \FunctionTok{factor}\NormalTok{(data\_tidy\_air\_quality}\SpecialCharTok{$}\NormalTok{industrial\_activity, }
                                   \AttributeTok{levels =} \FunctionTok{c}\NormalTok{(}\StringTok{"None"}\NormalTok{,}\StringTok{"Low"}\NormalTok{, }\StringTok{"Moderate"}\NormalTok{, }\StringTok{"High"}\NormalTok{))  }\CommentTok{\# Adjust the levels according to your data}

\NormalTok{data\_tidy\_air\_quality}\SpecialCharTok{$}\NormalTok{day\_of\_week }\OtherTok{\textless{}{-}} \FunctionTok{factor}\NormalTok{(data\_tidy\_air\_quality}\SpecialCharTok{$}\NormalTok{day\_of\_week, }
                           \AttributeTok{levels =} \FunctionTok{c}\NormalTok{(}\StringTok{"Monday"}\NormalTok{, }\StringTok{"Tuesday"}\NormalTok{, }\StringTok{"Wednesday"}\NormalTok{, }
                                      \StringTok{"Thursday"}\NormalTok{, }\StringTok{"Friday"}\NormalTok{, }\StringTok{"Saturday"}\NormalTok{, }\StringTok{"Sunday"}\NormalTok{))}

\NormalTok{data\_tidy\_air\_quality}\SpecialCharTok{$}\NormalTok{holiday }\OtherTok{\textless{}{-}} \FunctionTok{factor}\NormalTok{(data\_tidy\_air\_quality}\SpecialCharTok{$}\NormalTok{holiday, }
                           \AttributeTok{levels =} \FunctionTok{c}\NormalTok{(}\StringTok{"Yes"}\NormalTok{, }\StringTok{"No"}\NormalTok{))}

\NormalTok{categorical\_vars }\OtherTok{\textless{}{-}} \FunctionTok{names}\NormalTok{(data\_tidy\_air\_quality)[}\FunctionTok{sapply}\NormalTok{(data\_tidy\_air\_quality, is.factor)]}


\ControlFlowTok{for}\NormalTok{ (var }\ControlFlowTok{in}\NormalTok{ categorical\_vars) \{}
\NormalTok{  plt}\OtherTok{\textless{}{-}} \FunctionTok{ggplot}\NormalTok{(data\_tidy\_air\_quality, }\FunctionTok{aes\_string}\NormalTok{(}\AttributeTok{x =}\NormalTok{ var, }\AttributeTok{y =} \StringTok{"particulate\_matter"}\NormalTok{)) }\SpecialCharTok{+}
    \FunctionTok{geom\_boxplot}\NormalTok{() }\SpecialCharTok{+}
    \FunctionTok{labs}\NormalTok{(}\AttributeTok{title =} \FunctionTok{paste}\NormalTok{(}\StringTok{"Particulate Matter vs"}\NormalTok{, var),}
         \AttributeTok{x =}\NormalTok{ var,}
         \AttributeTok{y =} \StringTok{"Particulate Matter"}\NormalTok{) }\SpecialCharTok{+}
    \FunctionTok{theme\_minimal}\NormalTok{() }
    
  
  \FunctionTok{print}\NormalTok{(plt)  }\CommentTok{\# Print the plot}
\NormalTok{\}}
\end{Highlighting}
\end{Shaded}

\begin{verbatim}
## Warning: `aes_string()` was deprecated in ggplot2 3.0.0.
## i Please use tidy evaluation idioms with `aes()`.
## i See also `vignette("ggplot2-in-packages")` for more information.
## This warning is displayed once every 8 hours.
## Call `lifecycle::last_lifecycle_warnings()` to see where this warning was
## generated.
\end{verbatim}

\includegraphics{Report_files/figure-latex/unnamed-chunk-2-1.pdf}
\includegraphics{Report_files/figure-latex/unnamed-chunk-2-2.pdf}
\includegraphics{Report_files/figure-latex/unnamed-chunk-2-3.pdf}
tabular representation of relationship between categorial variables

\begin{Shaded}
\begin{Highlighting}[]
\ControlFlowTok{for}\NormalTok{ (i }\ControlFlowTok{in} \DecValTok{1}\SpecialCharTok{:}\NormalTok{(}\FunctionTok{length}\NormalTok{(categorical\_vars)}\SpecialCharTok{{-}}\DecValTok{1}\NormalTok{)) \{}
  \ControlFlowTok{for}\NormalTok{ (j }\ControlFlowTok{in}\NormalTok{ (i}\SpecialCharTok{+}\DecValTok{1}\NormalTok{)}\SpecialCharTok{:}\FunctionTok{length}\NormalTok{(categorical\_vars)) \{}
    \FunctionTok{cat}\NormalTok{(}\StringTok{"Contingency Table for"}\NormalTok{, categorical\_vars[i], }\StringTok{"and"}\NormalTok{, categorical\_vars[j], }\StringTok{"}\SpecialCharTok{\textbackslash{}n}\StringTok{"}\NormalTok{)}
    \FunctionTok{print}\NormalTok{(}\FunctionTok{table}\NormalTok{(data\_tidy\_air\_quality[[categorical\_vars[i]]], data\_tidy\_air\_quality[[categorical\_vars[j]]]))}
    \FunctionTok{cat}\NormalTok{(}\StringTok{"}\SpecialCharTok{\textbackslash{}n}\StringTok{"}\NormalTok{)}
\NormalTok{  \}}
\NormalTok{\}}
\end{Highlighting}
\end{Shaded}

\begin{verbatim}
## Contingency Table for industrial_activity and day_of_week 
##           
##            Monday Tuesday Wednesday Thursday Friday Saturday Sunday
##   None          2       0         3        3      2        0      4
##   Low           5       6         4        7      6        9      4
##   Moderate      4       4        10        8      6        4      3
##   High         11       7         9        5      8       10      6
## 
## Contingency Table for industrial_activity and holiday 
##           
##            Yes No
##   None       5  9
##   Low       17 24
##   Moderate   9 30
##   High      21 35
## 
## Contingency Table for day_of_week and holiday 
##            
##             Yes No
##   Monday      1 21
##   Tuesday     1 16
##   Wednesday   3 23
##   Thursday    4 19
##   Friday      3 19
##   Saturday   23  0
##   Sunday     17  0
\end{verbatim}

visual representation of relationship between categorial variables

\begin{Shaded}
\begin{Highlighting}[]
\ControlFlowTok{for}\NormalTok{ (i }\ControlFlowTok{in} \DecValTok{1}\SpecialCharTok{:}\NormalTok{(}\FunctionTok{length}\NormalTok{(categorical\_vars) }\SpecialCharTok{{-}} \DecValTok{1}\NormalTok{)) \{}
  \ControlFlowTok{for}\NormalTok{ (j }\ControlFlowTok{in}\NormalTok{ (i }\SpecialCharTok{+} \DecValTok{1}\NormalTok{)}\SpecialCharTok{:}\FunctionTok{length}\NormalTok{(categorical\_vars)) \{}
    \CommentTok{\# Create the plot}
\NormalTok{    p }\OtherTok{\textless{}{-}} \FunctionTok{ggplot}\NormalTok{(data\_tidy\_air\_quality, }\FunctionTok{aes\_string}\NormalTok{(}\AttributeTok{x =}\NormalTok{ categorical\_vars[i], }\AttributeTok{fill =}\NormalTok{ categorical\_vars[j])) }\SpecialCharTok{+}
      \FunctionTok{geom\_bar}\NormalTok{(}\AttributeTok{position =} \StringTok{"fill"}\NormalTok{) }\SpecialCharTok{+}  \CommentTok{\# Use "fill" to make it a stacked bar chart (proportions)}
      \FunctionTok{labs}\NormalTok{(}\AttributeTok{title =} \FunctionTok{paste}\NormalTok{(}\StringTok{"Stacked Bar Chart of"}\NormalTok{, categorical\_vars[i], }\StringTok{"and"}\NormalTok{, categorical\_vars[j]),}
           \AttributeTok{x =}\NormalTok{ categorical\_vars[i],}
           \AttributeTok{y =} \StringTok{"Proportion"}\NormalTok{) }\SpecialCharTok{+}
      \FunctionTok{theme\_minimal}\NormalTok{() }\SpecialCharTok{+}
      \FunctionTok{theme}\NormalTok{(}\AttributeTok{axis.text.x =} \FunctionTok{element\_text}\NormalTok{(}\AttributeTok{angle =} \DecValTok{45}\NormalTok{, }\AttributeTok{hjust =} \DecValTok{1}\NormalTok{))}
    
    \CommentTok{\# Print the plot}
    \FunctionTok{print}\NormalTok{(p)}
\NormalTok{  \}}
\NormalTok{\}}
\end{Highlighting}
\end{Shaded}

\includegraphics{Report_files/figure-latex/unnamed-chunk-4-1.pdf}
\includegraphics{Report_files/figure-latex/unnamed-chunk-4-2.pdf}
\includegraphics{Report_files/figure-latex/unnamed-chunk-4-3.pdf}
comments\\
distribution characterisitcs\\
The distribution of particulate matter levels is generally right-skewed,
indicating that a small number of observations have significantly high
levels of particulate matter while most observations are clustered at
lower levels. The presence of outliers suggests variations in local
conditions affecting air quality.\\
Observed Relationships\\
1. Traffic Density: A positive correlation exists between particulate
matter levels and traffic density, suggesting that areas with higher
vehicle traffic tend to experience elevated levels of particulate
matter.\\
2. Urban Greenery: A negative trend is observed, where higher urban
greenery correlates with lower particulate matter, indicating that
vegetation may help mitigate air pollution.\\
3. Temperature and Wind Speed: No strong relationship was identified
between particulate matter and temperature. However, there is a slight
negative correlation with wind speed, indicating that higher wind speeds
may help disperse particulate matter.

Potential Collinearity\\
Some potential collinearity is observed among the explanatory variables,
particularly between traffic density and urban greenery. High traffic
areas often have less vegetation, leading to a relationship that may
confound the analysis. Additionally, temperature and wind speed may also
exhibit collinearity, as changes in one could affect the other.

\hypertarget{section-3}{%
\section{Section 3}\label{section-3}}

simple linear regression

\begin{Shaded}
\begin{Highlighting}[]
\NormalTok{X }\OtherTok{\textless{}{-}} \FunctionTok{cbind}\NormalTok{(}\DecValTok{1}\NormalTok{,data\_tidy\_air\_quality}\SpecialCharTok{$}\NormalTok{traffic\_density)}

\NormalTok{Y }\OtherTok{\textless{}{-}}\NormalTok{data\_tidy\_air\_quality}\SpecialCharTok{$}\NormalTok{particulate\_matter}
\NormalTok{bhat }\OtherTok{\textless{}{-}} \FunctionTok{solve}\NormalTok{(}\FunctionTok{t}\NormalTok{(X) }\SpecialCharTok{\%*\%}\NormalTok{ X) }\SpecialCharTok{\%*\%} \FunctionTok{t}\NormalTok{(X) }\SpecialCharTok{\%*\%}\NormalTok{ Y}

\NormalTok{Cmat }\OtherTok{\textless{}{-}} \FunctionTok{solve}\NormalTok{(}\FunctionTok{t}\NormalTok{(X) }\SpecialCharTok{\%*\%}\NormalTok{ X)}

\NormalTok{k }\OtherTok{\textless{}{-}} \FunctionTok{ncol}\NormalTok{(X)}
\NormalTok{rss }\OtherTok{\textless{}{-}} \FunctionTok{t}\NormalTok{(Y }\SpecialCharTok{{-}}\NormalTok{ X }\SpecialCharTok{\%*\%}\NormalTok{ bhat) }\SpecialCharTok{\%*\%}\NormalTok{ (Y }\SpecialCharTok{{-}}\NormalTok{ X }\SpecialCharTok{\%*\%}\NormalTok{ bhat)}
\CommentTok{\# Calculate s2 = RSS/(n{-}k)}
\NormalTok{s2 }\OtherTok{\textless{}{-}} \FunctionTok{as.numeric}\NormalTok{((rss)}\SpecialCharTok{/}\DecValTok{148}\NormalTok{)}
\NormalTok{s2}
\end{Highlighting}
\end{Shaded}

\begin{verbatim}
## [1] 143.5745
\end{verbatim}

\begin{Shaded}
\begin{Highlighting}[]
\NormalTok{c\_ii }\OtherTok{\textless{}{-}} \FunctionTok{diag}\NormalTok{(Cmat)}

\NormalTok{std.error }\OtherTok{\textless{}{-}} \FunctionTok{sqrt}\NormalTok{(s2 }\SpecialCharTok{*}\NormalTok{ c\_ii)}
\NormalTok{std.error}
\end{Highlighting}
\end{Shaded}

\begin{verbatim}
## [1] 20.37801682  0.04065266
\end{verbatim}

\begin{Shaded}
\begin{Highlighting}[]
\NormalTok{mod1}\OtherTok{\textless{}{-}}\FunctionTok{lm}\NormalTok{(data\_tidy\_air\_quality}\SpecialCharTok{$}\NormalTok{particulate\_matter }\SpecialCharTok{\textasciitilde{}}\NormalTok{ data\_tidy\_air\_quality}\SpecialCharTok{$}\NormalTok{traffic\_density, }\AttributeTok{data =}\NormalTok{ data\_tidy\_air\_quality)}

\FunctionTok{summary}\NormalTok{(mod1)}
\end{Highlighting}
\end{Shaded}

\begin{verbatim}
## 
## Call:
## lm(formula = data_tidy_air_quality$particulate_matter ~ data_tidy_air_quality$traffic_density, 
##     data = data_tidy_air_quality)
## 
## Residuals:
##     Min      1Q  Median      3Q     Max 
## -28.332  -7.561  -1.050   6.110  35.243 
## 
## Coefficients:
##                                       Estimate Std. Error t value Pr(>|t|)  
## (Intercept)                           18.11537   20.37802   0.889   0.3755  
## data_tidy_air_quality$traffic_density  0.08400    0.04065   2.066   0.0406 *
## ---
## Signif. codes:  0 '***' 0.001 '**' 0.01 '*' 0.05 '.' 0.1 ' ' 1
## 
## Residual standard error: 11.98 on 148 degrees of freedom
## Multiple R-squared:  0.02804,    Adjusted R-squared:  0.02147 
## F-statistic: 4.269 on 1 and 148 DF,  p-value: 0.04055
\end{verbatim}

hypthesis test

\begin{Shaded}
\begin{Highlighting}[]
\CommentTok{\# Summary of ANOVA results}
\FunctionTok{summary}\NormalTok{(}\FunctionTok{aov}\NormalTok{(particulate\_matter }\SpecialCharTok{\textasciitilde{}}\NormalTok{ industrial\_activity, }\AttributeTok{data =}\NormalTok{ data\_tidy\_air\_quality))}
\end{Highlighting}
\end{Shaded}

\begin{verbatim}
##                      Df Sum Sq Mean Sq F value Pr(>F)   
## industrial_activity   3   2182   727.3   5.396 0.0015 **
## Residuals           146  19680   134.8                  
## ---
## Signif. codes:  0 '***' 0.001 '**' 0.01 '*' 0.05 '.' 0.1 ' ' 1
\end{verbatim}

\begin{Shaded}
\begin{Highlighting}[]
\CommentTok{\# Calculate F{-}statistic and p{-}value manually}
\NormalTok{group\_means }\OtherTok{\textless{}{-}} \FunctionTok{tapply}\NormalTok{(data\_tidy\_air\_quality}\SpecialCharTok{$}\NormalTok{particulate\_matter, data\_tidy\_air\_quality}\SpecialCharTok{$}\NormalTok{industrial\_activity, mean)}
\NormalTok{overall\_mean }\OtherTok{\textless{}{-}} \FunctionTok{mean}\NormalTok{(data\_tidy\_air\_quality}\SpecialCharTok{$}\NormalTok{particulate\_matter)}

\CommentTok{\# Calculate SST}
\NormalTok{SST }\OtherTok{\textless{}{-}} \FunctionTok{sum}\NormalTok{((data\_tidy\_air\_quality}\SpecialCharTok{$}\NormalTok{particulate\_matter }\SpecialCharTok{{-}}\NormalTok{ overall\_mean)}\SpecialCharTok{\^{}}\DecValTok{2}\NormalTok{)}

\CommentTok{\# Calculate SSB}
\NormalTok{n }\OtherTok{\textless{}{-}} \FunctionTok{table}\NormalTok{(data\_tidy\_air\_quality}\SpecialCharTok{$}\NormalTok{industrial\_activity)}
\NormalTok{SStreatment }\OtherTok{\textless{}{-}} \FunctionTok{sum}\NormalTok{(n }\SpecialCharTok{*}\NormalTok{ (group\_means }\SpecialCharTok{{-}}\NormalTok{ overall\_mean)}\SpecialCharTok{\^{}}\DecValTok{2}\NormalTok{)}

\CommentTok{\# Calculate SSW}
\NormalTok{group\_means\_vector }\OtherTok{\textless{}{-}} \FunctionTok{unlist}\NormalTok{(}\FunctionTok{tapply}\NormalTok{(data\_tidy\_air\_quality}\SpecialCharTok{$}\NormalTok{particulate\_matter, data\_tidy\_air\_quality}\SpecialCharTok{$}\NormalTok{industrial\_activity, mean)[data\_tidy\_air\_quality}\SpecialCharTok{$}\NormalTok{industrial\_activity])}
\NormalTok{SSerror }\OtherTok{\textless{}{-}} \FunctionTok{sum}\NormalTok{((data\_tidy\_air\_quality}\SpecialCharTok{$}\NormalTok{particulate\_matter }\SpecialCharTok{{-}}\NormalTok{ group\_means\_vector)}\SpecialCharTok{\^{}}\DecValTok{2}\NormalTok{)}

\CommentTok{\# Calculate degrees of freedom}
\NormalTok{k }\OtherTok{\textless{}{-}} \FunctionTok{length}\NormalTok{(}\FunctionTok{unique}\NormalTok{(data\_tidy\_air\_quality}\SpecialCharTok{$}\NormalTok{industrial\_activity))}
\NormalTok{N }\OtherTok{\textless{}{-}} \FunctionTok{nrow}\NormalTok{(data)}
\NormalTok{DFtreatment }\OtherTok{\textless{}{-}}\NormalTok{ k }\SpecialCharTok{{-}} \DecValTok{1}
\NormalTok{DFerror }\OtherTok{\textless{}{-}} \DecValTok{150} \SpecialCharTok{{-}}\NormalTok{ k}

\CommentTok{\# Calculate Mean Squares}
\NormalTok{MStreatment }\OtherTok{\textless{}{-}}\NormalTok{ SStreatment }\SpecialCharTok{/}\NormalTok{ DFtreatment}
\NormalTok{MSerror }\OtherTok{\textless{}{-}}\NormalTok{ SSerror }\SpecialCharTok{/}\NormalTok{ DFerror}


\CommentTok{\# Calculate F{-}statistic}
\NormalTok{F\_statistic }\OtherTok{\textless{}{-}}\NormalTok{ MStreatment}\SpecialCharTok{/}\NormalTok{MSerror}

\CommentTok{\# Output F{-}statistic}
\NormalTok{F\_statistic}
\end{Highlighting}
\end{Shaded}

\begin{verbatim}
## [1] 5.395959
\end{verbatim}

\begin{Shaded}
\begin{Highlighting}[]
\CommentTok{\# Calculate p{-}value}
\NormalTok{p\_value }\OtherTok{\textless{}{-}} \FunctionTok{pf}\NormalTok{(F\_statistic, DFtreatment, DFerror, }\AttributeTok{lower.tail =} \ConstantTok{FALSE}\NormalTok{)}
\NormalTok{p\_value}
\end{Highlighting}
\end{Shaded}

\begin{verbatim}
## [1] 0.001502236
\end{verbatim}

\newpage

\hypertarget{question-4}{%
\section{Question 4}\label{question-4}}

\begingroup\fontsize{12}{14}\selectfont

\begin{longtable}[t]{lccc}
\caption{\label{tab:unnamed-chunk-7}Confidence Interval for each Coefficients}\\
\toprule
 & 2.5 \% & Estimate & 97.5 \%\\
\midrule
\addlinespace[0.3em]
\multicolumn{4}{l}{\textbf{Intercept}}\\
\hspace{1em}(Intercept) & -21.0568 & 13.7937 & 48.6442\\
\addlinespace[0.3em]
\multicolumn{4}{l}{\textbf{Traffic Density}}\\
\hspace{1em}traffic\_density & 0.0155 & 0.0799 & 0.1444\\
\addlinespace[0.3em]
\multicolumn{4}{l}{\textbf{Industrial Activity}}\\
\hspace{1em}industrial\_activityLow & -3.1721 & 2.6589 & 8.4900\\
\hspace{1em}industrial\_activityModerate & 0.6047 & 6.4545 & 12.3043\\
\hspace{1em}industrial\_activityHigh & -0.2503 & 5.3652 & 10.9806\\
\addlinespace[0.3em]
\multicolumn{4}{l}{\textbf{Natural Factors}}\\
\hspace{1em}temperature & -1.1521 & -0.2815 & 0.5891\\
\hspace{1em}humidity & -0.1111 & 0.1926 & 0.4962\\
\hspace{1em}wind\_speed & -0.8040 & 0.0193 & 0.8426\\
\hspace{1em}temperature:humidity & -0.0088 & 0.0061 & 0.0209\\
\addlinespace[0.3em]
\multicolumn{4}{l}{\textbf{Day of Week}}\\
\hspace{1em}day\_of\_weekTuesday & -5.9877 & 0.0133 & 6.0142\\
\hspace{1em}day\_of\_weekWednesday & -5.3501 & 0.1565 & 5.6630\\
\hspace{1em}day\_of\_weekThursday & -5.5367 & 0.1662 & 5.8690\\
\hspace{1em}day\_of\_weekFriday & -8.0602 & -2.4221 & 3.2161\\
\hspace{1em}day\_of\_weekSaturday & -12.3605 & -4.4832 & 3.3940\\
\hspace{1em}day\_of\_weekSunday & -10.2167 & -2.0885 & 6.0396\\
\addlinespace[0.3em]
\multicolumn{4}{l}{\textbf{Holiday}}\\
\hspace{1em}holidayNo & -6.7151 & -0.9961 & 4.7228\\
\addlinespace[0.3em]
\multicolumn{4}{l}{\textbf{Urban Greenery}}\\
\hspace{1em}urban\_greenery & -0.4142 & -0.2954 & -0.1766\\
\bottomrule
\end{longtable}
\endgroup{}

\hypertarget{hypothesis-testing}{%
\subsubsection{Hypothesis Testing}\label{hypothesis-testing}}

We'd like to perform hypothesis tests on the following variables:
Temperature, Humidity, Industrial Levels, and Day of Week.

We'll start by examining whether Temperature has an effect on the
concentration of Particulate Matter. We'll use the following We use the
following set of hypothesis: \begin{align}
\[&H_0: \beta_{temp} = \beta_{hum:temp} = 0 \\ 
&H_A: \beta_{temp} \neq 0 \text{ and } \beta_{hum:temp} \neq 0\]
\end{align} This can be done by comparing the restricted and un
restricted model: \[Y_R = \beta_0 + \beta_{traffic}X + \]

\begin{Shaded}
\begin{Highlighting}[]
\NormalTok{model\_unrestricted }\OtherTok{\textless{}{-}} \FunctionTok{lm}\NormalTok{(particulate\_matter }\SpecialCharTok{\textasciitilde{}}\NormalTok{ . }\SpecialCharTok{+} 
\NormalTok{                         temperature}\SpecialCharTok{:}\NormalTok{humidity,}
                         \AttributeTok{data=}\NormalTok{data\_tidy\_air\_quality)}
\NormalTok{model\_restricted }\OtherTok{\textless{}{-}} \FunctionTok{update}\NormalTok{(model\_unrestricted, .}\SpecialCharTok{\textasciitilde{}}\NormalTok{.}
                           \SpecialCharTok{{-}}\NormalTok{ temperature}
                           \SpecialCharTok{{-}}\NormalTok{ temperature}\SpecialCharTok{:}\NormalTok{humidity)}
\FunctionTok{kable}\NormalTok{(}\FunctionTok{anova}\NormalTok{(model\_unrestricted, model\_restricted))}
\end{Highlighting}
\end{Shaded}

\begin{longtable}[]{@{}rrrrrr@{}}
\toprule\noalign{}
Res.Df & RSS & Df & Sum of Sq & F & Pr(\textgreater F) \\
\midrule\noalign{}
\endhead
\bottomrule\noalign{}
\endlastfoot
133 & 11032.02 & NA & NA & NA & NA \\
135 & 11095.83 & -2 & -63.80113 & 0.3845872 & 0.6814861 \\
\end{longtable}

Using the anova function in R, we compare the two models with F test,
yielding a P value 0.6815, suggesting that temperature doesn't have a
significant effect on the concentration of particular matter.






\end{document}
